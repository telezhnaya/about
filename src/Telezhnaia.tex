\documentclass[11pt]{res} % default is 10 pt
%\usepackage{helvetica} % uses helvetica postscript font (download helvetica.sty)
%\usepackage{newcent}   % uses new century schoolbook postscript font 
%\setlength{\textheight}{9.5in} % increase text height to fit resume on 1 page
\newsectionwidth{0pt}  % So the text is not indented under section headings

\usepackage{color}
\usepackage{setspace}
\usepackage{fontawesome}
\usepackage[hidelinks]{hyperref}

% lastpage does not work with hyperref :-(
%\usepackage{lastpage}
\usepackage{fancyhdr}
\pagestyle{fancy}
% My pagestyle is fancy because I want to put footers, but I remove everything else from this fancy style
\fancyhf{}
\renewcommand{\headrulewidth}{0pt}

% Redefinition, symbol included in link:
\let\orighref\href
\renewcommand{\href}[2]{\orighref{#1}{#2\,\faExternalLink}}

\tolerance=1
\emergencystretch=\maxdimen
\hyphenpenalty=10000
\hbadness=10000

\begin{document} 
 
\name{\Huge Olga Telezhnaia\\[20pt]} % the \\[12pt] adds a blank line after name

\address{April 2019 \\ Saint Petersburg, Russia}

\address{\href{mailto:olyatelezhnaya@gmail.com}{olyatelezhnaya@gmail.com} \\
\href{https://github.com/telezhnaya}{github.com/telezhnaya} \\ +7 905 287 27 73}
                                             
\begin{resume}

%\makebox[\linewidth]{\rule{\textwidth}{0.4pt}}

\section{\uppercase{Personal Projects}}
\begin{spacing}{0.3}
\textcolor[RGB]{220,220,220}{\rule{\linewidth}{0.4pt}} 
\end{spacing}
\textbf{\href{https://github.com/git/git/commits?author=telezhnaya}{Contributions to Git}} (C, Shell, Git)~--- Optimizing formatting system of Git (up to 3 times faster now), increased functionality of cat-file, for-each-ref, ls-remote etc, unifying codebase, decreasing memory leaks. \\ \\
\textbf{\href{https://github.com/telezhnaya/linux}{Scheduler in Linux Kernel}} (Linux, C, Git)~--- Created the prototype of block IO scheduler for groups of disks. It allows giving priority manually to one group of disks for read/write processes at high load points.

\section{\uppercase{Working Experience} (2.5+ years total)}
\begin{spacing}{0.3}
\textcolor[RGB]{220,220,220}{\rule{\linewidth}{0.4pt}} 
\end{spacing}
\vspace{0.2in}
   \begin{itemize} \itemsep -2pt  % reduce space between items
   \item {\textbf{Senior SWE at Security Code}} (Python, Flask, RabbitMQ, PostgreSQL, SQLAlchemy, Marshmallow)~--- Developing backend for web client of security tool. (September 2018 - now)
   \item {\textbf{TE at Microsoft}} (Python, C\#, ML, Linux, Azure)~--- Developed internal projects, shared expertise at conferences, wrote technical articles. (February 2017 - September 2018)
   \item {\textbf{SWE intern at Yandex}} (C++, Python, MapReduce, Bash, svn, regex)~--- Improved quality of search by geographical objects with help of linguistic features of Russian language. Added support for sides of the world in queries. (July – October 2016)
   \item \textbf{\href{https://github.com/ignatov/intellij-erlang/commits?author=telezhnaya}{SWE intern at JetBrains}} (Java, JUnit, Erlang, git)~--- Developed IntelliJ plugin for Erlang programming language. Created multiple inspections, autocompleting, links resolution requested by users. (July – September 2015)
 \end{itemize}

\section{\uppercase{Education}} 
\begin{spacing}{0.3}
\textcolor[RGB]{220,220,220}{\rule{\linewidth}{0.4pt}} 
\end{spacing}
\noindent \textbf{Peter the Great Saint-Petersburg Polytechnic University} \\
Bachelor of Software and Administration of Information Systems (2012---2016) \\
GPA $4.8/5.0$ \\ \\ 
\textbf{\href{https://compscicenter.ru/about/}{Computer Science Center}} \\
\href{https://compscicenter.ru/users/801/}{Software Engineering specialization} (2014---2016) \\
\textbf{Courses passed:} Algorithms and Data Structures (in 2 parts) $\cdot$ Fundamentals of Discrete Mathematics $\cdot$ Computer Architecture and OS $\cdot$ C++ (in 2 parts) $\cdot$ Image Analysis (in 2 parts) $\cdot$ Python $\cdot$ Parallel Programming $\cdot$ Android Development $\cdot$ Machine Learning (Course of Moscow Data Analysis School) \\ \\ 
\noindent \textbf{MOOCs:} \href{https://stepik.org/certificate/c97c48eb77cd3edbf2b4cb8d1624930ffe87cb7c.pdf}{Algorithms and Data Structures} $\cdot$ \href{https://www.coursera.org/learn/machine-learning}{Machine Learning by Andrew Ng, Stanford} $\cdot$ \href{https://www.coursera.org/learn/matematika-dlya-vseh}{Mathematics for everyone by Aleksey Savvateev}

\cfoot{1 of 2}
\null\newpage

\section{\uppercase{Social Impact}}
\begin{spacing}{0.3}
\textcolor[RGB]{220,220,220}{\rule{\linewidth}{0.4pt}} 
\end{spacing}
\textbf{\href{http://djangogirls.org/stpetersburg/}{Django Girls}}~--- Organized 2 events in Saint Petersburg, coached 3 new organizers and nearly 25 mentors, launching first Django Girls workshop in Kazan. Nearly 120 women started their journey in technologies after my workshops.\\ \\
\textbf{\href{https://git.github.io/SoC-2019-Ideas/}{GSoC 2019: Mentor in Git}}~--- Created project proposals, selected students. This summer, as active Git contributor, I will mentor one of GSoC students.\\ \\
\textbf{\href{https://www.youtube.com/programming_together/}{YouTube Channel}}~--- My videos help to empower other people to get into technologies. I do not have exact numbers, but letters in my public email said that 3 women entered technical University and 2 other women were selected to GSoC/Outreachy internship after watching my videos.\\ \\
\textbf{\href{https://dmkpress.com/catalog/computer/}{Books about IT}}~--- Reviewing books about technologies, writing proposals, editing translation, editing code examples so that my colleagues or studens or children (who does not know English still) could enjoy STEM books in Russian from around the world.\\ \\
\textbf{Teaching}~--- Teached Math and Programming for more than 5 years.\\ \\
\textbf{Other}~--- Took part in Hacktoberfest, MS Hackathon, Google Code Jam to I/O. I always visit as many conferences as I can to share my knowledge and find new friends. 

\textbf{My biggest passion is} to develop parts of large project. I enjoy my main SWE role, and I am happy to make social impact projects as a hobby.

\section{\uppercase{Preferred technologies \& concepts}}
\begin{spacing}{0.3}
\textcolor[RGB]{220,220,220}{\rule{\linewidth}{0.4pt}} 
\end{spacing}
C/C++ $\cdot$ Python $\cdot$ Git $\cdot$ Algorithms $\cdot$ Data Structures $\cdot$ Linux $\cdot$ OOP $\cdot$ Image Analysis $\cdot$ Machine Learning $\cdot$ Linguistics $\cdot$ SQL $\cdot$ Conferences (listener, speaker, organizer) $\cdot$ System Programming \\ \\
\textbf{Languages:} Russian native, English fluent

\cfoot{2 of 2}
\end{resume}
\end{document}
