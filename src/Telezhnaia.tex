\documentclass[11pt]{res} % default is 10 pt
%\usepackage{helvetica} % uses helvetica postscript font (download helvetica.sty)
%\usepackage{newcent}   % uses new century schoolbook postscript font 
\setlength{\textheight}{9.5in} % increase text height to fit resume on 1 page
\newsectionwidth{0pt}  % So the text is not indented under section headings

\usepackage{color}
\usepackage{setspace}
\usepackage{fontawesome}
\usepackage[hidelinks]{hyperref}

% Redefinition, symbol included in link:
\let\orighref\href
\renewcommand{\href}[2]{\orighref{#1}{#2\,\faExternalLink}}

\tolerance=1
\emergencystretch=\maxdimen
\hyphenpenalty=10000
\hbadness=10000
\begin{document} 

 
\name{\Huge Olga Telezhnaia\\[20pt]} % the \\[12pt] adds a blank line after name

\address{November 2018 \\ Saint Petersburg, Russia}

\address{\href{mailto:olyatelezhnaya@gmail.com}{olyatelezhnaya@gmail.com} \\
\href{https://github.com/telezhnaya}{github.com/telezhnaya} \\ +7 905 287 27 73}
                                             
\begin{resume}

%\makebox[\linewidth]{\rule{\textwidth}{0.4pt}}

\section{\uppercase{Personal Projects}}
\begin{spacing}{0.3}
\textcolor[RGB]{220,220,220}{\rule{\linewidth}{0.4pt}} 
\end{spacing}
\textbf{\href{https://github.com/git/git/commits?author=telezhnaya}{Contributions to Git}} (C, Shell, Git)~--- Optimize formatting system of Git (up to 3 times faster), increase functionality of cat-file, for-each-ref, ls-remote etc, unify codebase, decrease memory leaks. \\ \\
\textbf{\href{https://github.com/telezhnaya/linux}{Scheduler in Linux Kernel}} (Linux, C, Git)~--- Created the prototype of block IO scheduler for groups of disks. It allows giving priority manually to one group of disks for read/write processes at high load points.

\section{\uppercase{Working Experience} (2+ years total)}
\begin{spacing}{0.3}
\textcolor[RGB]{220,220,220}{\rule{\linewidth}{0.4pt}} 
\end{spacing}
\vspace{0.2in}
   \begin{itemize} \itemsep -2pt  % reduce space between items
   \item {\textbf{Senior SWE at Security Code}} (Python, Flask, RabbitMQ, PostgreSQL, SQLAlchemy, Marshmallow)~--- Backend web development. (September 2018 - now)
   \item {\textbf{TE at Microsoft}} (Python, C\#, Machine Learning, Linux, Azure)~--- Developed internal projects, spoke at conferences, wrote technical articles. (February 2017 - September 2018)
   \item {\textbf{SWE intern at Yandex}} (C++, Python, MapReduce, Bash, svn, regex)~--- Improved quality of search by geographical objects with help of linguistic features of Russian language. Added support for sides of the world. (July – October 2016)
   \item \textbf{\href{https://github.com/ignatov/intellij-erlang/commits?author=telezhnaya}{SWE intern at JetBrains}} (Java, JUnit, Erlang, git)~--- Developed IntelliJ plugin for Erlang programming language. Created multiple inspections, autocompleting, links resolution requested by users. (July – September 2015)
 \end{itemize}

\section{\uppercase{Education}} 
\begin{spacing}{0.3}
\textcolor[RGB]{220,220,220}{\rule{\linewidth}{0.4pt}} 
\end{spacing}
\noindent \textbf{Peter the Great Saint-Petersburg Polytechnic University} \\
Bachelor of Software and Administration of Information Systems (2012---2016) \\
GPA $4.8/5.0$ \\ \\ 
\textbf{\href{https://compscicenter.ru/about/}{Computer Science Center}} \\
\href{https://compscicenter.ru/users/801/}{Software Engineering specialization} (2014---2016) \\
\textbf{Courses passed:} Algorithms and Data Structures (in 2 parts) $\cdot$ Fundamentals of Discrete Mathematics $\cdot$ Computer Architecture and OS $\cdot$ C++ (in 2 parts) $\cdot$ Image Analysis (in 2 parts) $\cdot$ Python $\cdot$ Parallel Programming $\cdot$ Android Development $\cdot$ Machine Learning (Course of Moscow Data Analysis School) \\ \\ 
\noindent \textbf{MOOCs:} \href{https://stepik.org/certificate/c97c48eb77cd3edbf2b4cb8d1624930ffe87cb7c.pdf}{Algorithms and Data Structures} $\cdot$ \href{https://www.coursera.org/learn/machine-learning}{Machine Learning by Andrew Ng, Stanford} $\cdot$ \href{https://www.coursera.org/learn/matematika-dlya-vseh}{Mathematics for everyone by Aleksey Savvateev}

\section{\uppercase{Preferred technologies \& concepts}}
\begin{spacing}{0.3}
\textcolor[RGB]{220,220,220}{\rule{\linewidth}{0.4pt}} 
\end{spacing}
C/C++ $\cdot$ Python $\cdot$ Git $\cdot$ Algorithms $\cdot$ Data Structures $\cdot$ Linux $\cdot$ OOP $\cdot$ Image Analysis $\cdot$ Machine Learning $\cdot$ Linguistics $\cdot$ SQL $\cdot$ Conferences (listener, speaker, organizer) $\cdot$ System Programming \\
\textbf{Languages:} Russian native, English fluent

\end{resume}
\end{document}
